% !TEX root = ../tex/thesis.tex

% This makes Figure A instead of Figure 1
\renewcommand{\thefigure}{\Alph{figure}}

% This resets the figure counter
\setcounter{figure}{0}

% reset the footnote counter
\setcounter{footnote}{0}

% Set language to Dutch for correct word-breaks.
% It also changes the Figure into Figuur, etc.
%\renewcommand\chapterautorefname{hoofdstuk}%
\cleardoublepage

\chapter{Summary}

The \spitzer Space Telescope was decommissioned in January 2020, leaving behind a legacy of rich atmospheric studies of individual exoplanets. This thesis summarizes the huge efforts of \spitzer in the field of exoplanet science over the last 15 years. Using several hundreds of hours of near-infrared \spitzerIRAC observations in emission and transmission, we measured and characterized the atmospheres of more than 100 exoplanets in total. We gained a deep understanding of the complex instrumental systematics involved in analyzing such data and we pinned down the important physical processes required in atmospheric models.

In Chapter \ref{transits} we present our \spitzerIRAC data reduction pipeline. This pipeline implements pixel level decorrelation for correcting the strong systematics arising from the intrapixel sensitivity of the IRAC detectors. We used this procedure to analyze 70 photometric lightcurves of 33 transiting planets uniformly. We then augmented this sample with 16 previously published exoplanets, resulting in a total of 49 exoplanets with transmission measurements at 3.6 and 4.5\um.

We compare the survey of exoplanet transit depths to a grid of 1-D radiative convective equilibrium forward models. Our initial grid of models stands on two common assumptions when modelling exoplanet atmospheres. First, is that the atmosphere is in chemical equilibrium. This means that the abundance of different molecular species can be determined from the temperature, pressure, and global chemical composition of the atmosphere. Second, it is commonly assumed that the elemental chemical composition of a typical atmosphere is equivalent to that of the Sun (solar composition). Under these assumptions, models can predict the relative abundance of elemental and molecular species as a function of planet temperature and pressure probed by transmission spectroscopy. In particular, a transition from methane to carbon monoxide is expected at around 1000K as we sample from the coolest close-in giant exoplanets to the hottest hot-Jupiters. We demonstrated this transition with our grid of equilibrium chemistry solar composition forward atmospheric models. However, when comparing our models to our observations we found that the models overestimate methane levels for the coolest planets. With 13 planets less than 1000K, we obtained a strong statistical confirmation ($7.5~\sigma$) of the lack of methane in the atmospheres of gas giant exoplanets for the first time ever.

Building on this, we expanded our grid of models from equilibrium chemistry and solar composition to include disequilibrium chemistry as characterized by vertical mixing (via an eddy diffusion coefficient, $K_{zz} = 0$ - $10^{12}$~\cmcms) and with two different chemical compositions (1x and 30x solar). By comparing these models to the observations we were able to show that the lack of methane in the cool planets can be partially explained with models of higher metallicity (30x solar) and rule out the models with 1x solar composition with >$3~\sigma$ confidence. Previous studies have suggested that there is a mass-metallicity relation in the bulk composition of transiting gas giant planets, whereby the less massive planets tend to be more metal rich. Our finding supports the extrapolation of this trend from bulk metallicities to atmospheric metallicities.

Additionally, we found that these coolest planets (<1000K) favor the models with low amounts of vertical mixing ($K_{zz} = 10^8$~\cmcms). So not only do these cool, less massive close-in gas giant planets have statistically more metal-rich atmospheres than their hotter counterparts, they also exhibit less vigorous vertical mixing. On the other hand, we found that the hottest planets (>1000K) are best explained by 1x solar metallicity and high vertical mixing models ($K_{zz}= 10^{12}$~\cmcms). Higher levels of vertical mixing in hotter atmospheres has been predicted theoretically and observed in the atmospheres of brown dwarfs. Our result supports both of these studies.

In our survey of transiting planets, we did not find any obvious trends in the transmission metric with increasing temperature for the ultra-hot planets, but we only have a few ultra-hot Jupiters in our sample. However, in Chapter \ref{eclipses}, we expanded our exploration to the dayside emission of exoplanets. We used a survey of 78 planets with secondary eclipses to explore how the abundance of \ce{CH4} and CO manifests in the dayside. This survey contained twice as many hot planets compared with the transmission survey, and only a few of which fall within the expected methane-dominated temperature range. Furthermore, emission photometry probes deeper in the atmosphere (~1-10 bar pressure levels) than transmission photometry (~1mbar pressure level), reaching levels of higher temperatures and likely low methane abundance.

We examined this sample of planets by calculating the 3.6 and 4.5~\um~ brightness temperatures of the planetary dayside. We also defined a metric, called deviation from the blackbody, which measures the emission or absorption of the 4.5~\um relative to the 3.6~\um~ bandpass. The 4.5~\um~ probes the CO feature and the 3.6~\um~ probes close to the continuum due to the lack of methane in the hotter planets. Therefore, this metric allows us to probe the temperature pressure profile by measuring the strength of the relative CO emission or absorption. We found a transition in the deviation from the blackbody between the hot and the ultra-hot Jupiters at around 1700K, where the hotter planets appeared to have a stronger CO feature in emission. We explored what the origins of this might be by comparing the result to a new grid of self-consistent 1D radiative and convective models varying metallicity, carbon to oxygen ratio (C/O), surface gravity, and stellar effective temperature, making sure to incorporate the relevant physics for temperature inversions to form. The data is in remarkable agreement with these models. We propose that the transition between hot and ultra-hot Jupiters is statistical evidence of temperature inversions in the hottest planets, in addition to the expected Planck function shift.

Chapter \ref{eclipses} highlights the crucial importance of careful calculation of brightness temperatures and effective temperature. We found that either a failure to integrate over the \spitzer bandpass or approximating the star with a blackbody instead of a PHOENIX model when calculating the brightness temperatures can induce a bias in the results. This bias resulted in increasing the measured effective temperature of the planet compared to the equilibrium temperature prediction, and the bias was stronger for the planets around hotter stars. These disproportionately hotter effective temperatures in hotter exoplanets can be misinterpreted as a lower efficiency of redistribution in the hottest planets as is seen in previous studies. Another source of bias arises from the fact that the effective temperature is typically calculated by fitting a blackbody to the spectral energy distribution of the planet. However, there are just two photometric points, one of which has a strong CO emission feature (4.5~\um~) for the cases of ultra-hot Jupiters. In such a situation, calculating the effective temperature as a weighted mean of the two \spitzer brightness temperatures also biased the effective temperature results towards hotter temperatures. After correcting for all of these effects, we did not find a statistically significant trend in the effective temperature with equilibrium temperature. This finding does not support previous claims of lower redistribution efficiency with hotter planets. However, there remained a large scatter in the brightness temperatures of hotter planets compared to cooler planets, which suggested a range of different redistribution efficiencies for the hottest planets.

In Chapter \ref{w18b} we analyzed 10 archival secondary eclipses of the ultra-hot Jupiter WASP-18b and found periodic variability in the 4.5~\um~brightness of the planet in time. Using a sinusoidal model, we derived a variability period of 23.12 $\pm$ 1.66 days and a peak-to-trough amplitude of 456 $\pm$ 71~ppm, corresponding to $\sim$12\% variability. We discussed possible physical processes that could result in such variability: magnetic field coupling, variable wind speeds, clouds, changes in chemical composition, and we ruled out the hypothesis that this was due to stellar variability. Finally, we explored whether this could be detected with the current state-of-the-art instruments (HST, TESS) and found that these do not have the required precision and that we need to look towards future missions for follow-up of these variability measurements.

In Chapter \ref{TTVs} we analyzed 48 transits of some of the coolest planets observed with Spitzer/IRAC. We measured the transit times of six planets from three multi-planet systems (Kepler-9, Kepler-18 and Kepler-32) and compared these transit times to predictions made from Kepler observations. We found that the uncertainties were quite large on the \spitzer transits and so the results were consistent with predictions but not accurate enough to further constrain the models. Additionally, we analyzed two transits of the circumbinary planet, Kepler-16b. We use these transits in combination with archival Kepler data to constrain a photodynamical model and report updates on the orbital elements of the system.

The work presented in this thesis has the potential to serve as the benchmark for infrared studies in the future of exoplanet science, particularly with the upcoming James Webb Space Telescope. The trends observed and the effects described in the studies reported (vertical mixing, temperature inversions, clouds, variability, and complex dynamics) will be even more apparent with the increased precision, and important to include in future modelling efforts. Our research contributes to the understanding of planetary atmospheres in a broad context, thereby illuminating the issues of planet formation, evolution, and, ultimately, habitability.

% Reset language
\selectlanguage{english}
\renewcommand\chapterautorefname{chapter}%


%%% Local Variables:
%%% mode: latex
%%% TeX-master: "../thesis_renzo"
%%% End:
