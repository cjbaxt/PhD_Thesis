% !TEX root = ../tex/thesis.tex

% This makes Figure A instead of Figure 1
\renewcommand{\thefigure}{\Alph{figure}}

% This resets the figure counter
\setcounter{figure}{0}

% reset the footnote counter
\setcounter{footnote}{0}

% Set language to Dutch for correct word-breaks.
% It also changes the Figure into Figuur, etc.
\selectlanguage{dutch}
\renewcommand\chapterautorefname{hoofdstuk}%
\cleardoublepage

\chapter{Nederlandse Samenvatting}

De Spitzer ruimtetelescoop werd in januari 2020 buiten gebruik gesteld, een erfenis van rijke atmosferische studies van individuele exoplaneten achterlatend. Deze dissertatie vat de enorme inspanningen van Spitzer op het gebied van exoplaneetwetenschappen van de afgelopen 15 jaar samen. Met behulp van honderden uren aan nabije-infrarood Spitzer/IRAC waarnemingen in emissie en transmissie hebben we de atmosferen van in totaal meer dan 100 exoplaneten gemeten en gekarakteriseerd. We hebben een grondig inzicht gekregen in de complexe instrumentele systematiek die komt kijken bij het analyseren van dergelijke gegevens en we hebben de belangrijke fysische processen die nodig zijn in atmosferische modellen vastgelegd.

In Hoofdstuk \ref{transits} presenteren we onze Spitzer/IRAC datareductiepijplijn. Deze pijplijn implementeert decorrelatie op pixelniveau om de sterke systematiek te corrigeren die het gevolg is van de intrapixel gevoeligheid van de IRAC detectoren. We hebben deze procedure gebruikt om 70 fotometrische lichtkrommen van 33 planeten die voor hun ster langs gaan, op uniforme wijze te analyseren. Vervolgens hebben we deze groep uitgebreid met 16 eerder gepubliceerde exoplaneten, wat resulteerde in een totaal van 49 exoplaneten met transmissiemetingen bij 3,6 en 4,5\um.

We vergelijken het overzicht van exoplaneet overgangsdieptes met een raster van 1-D voorspellende modellen in radiatief convectief evenwicht. Ons oorspronkelijke raster van modellen berust op twee gangbare aannames bij het modelleren van exoplaneetatmosferen. De eerste is dat de atmosfeer in chemisch evenwicht is. Dit betekent dat de abundantie van verschillende moleculaire soorten kan worden bepaald uit de temperatuur, druk, en globale chemische samenstelling van de atmosfeer. Ten tweede wordt algemeen aangenomen dat de elementaire chemische samenstelling van een typische atmosfeer gelijk is aan die van de zon (zonne-samenstelling). Onder deze aannames kunnen modellen de relatieve abundantie van elementaire en moleculaire soorten voorspellen als functie van de temperatuur en druk van de planeet, gemeten met transmissie-spectroscopie. In het bijzonder wordt een overgang van methaan naar koolmonoxide verwacht bij ongeveer 1000 K, wanneer we van de koelste reuzenexoplaneten gaan die dicht bij hun ster staan, tot aan de heetste Hete Jupiters. Wij hebben deze overgang aangetoond met ons raster van atmosfeermodellen in chemisch evenwicht en zonne-samenstelling. Bij vergelijking van onze modellen met onze waarnemingen bleek echter dat de modellen de methaanconcentraties voor de koelste planeten overschatten. Met 13 planeten van minder dan 1000K hebben we voor het eerst een sterke statistische bevestiging ($7.5~\sigma$) van het ontbreken van methaan in de atmosferen van gasreuzen exoplaneten.

Hierop voortbouwend hebben we ons modelraster uitgebreid van evenwichtschemie en zonnesamenstelling met onevenwichtschemie zoals gekenmerkt door verticale menging (via een eddy-diffusiecoëfficiënt, $K_{zz} = 0$ - $10^{12}$~\cmcms) en met twee verschillende chemische samenstellingen (1x en 30x zonne-samenstelling). Door deze modellen te vergelijken met de waarnemingen konden we aantonen dat het gebrek aan methaan in de koele planeten gedeeltelijk verklaard kan worden met modellen van hogere metalliciteit (30x zonne-samenstelling) en konden we de modellen met 1x zonne-samenstelling uitsluiten met >$3~\sigma$. Eerdere studies hebben gesuggereerd dat er een relatie bestaat tussen de massa en de metalliciteit in de bulksamenstelling van gasreuzen, waarbij de minder massieve planeten over het algemeen rijker zijn aan metaal. Onze bevindingen ondersteunen de extrapolatie van deze trend van bulkmetalliciteiten naar atmosferische metalliciteiten.

Bovendien vonden we dat deze koelste planeten (<1000K) de voorkeur geven aan de modellen met weinig verticale menging ($K_{zz} = 10^8$~\cmcms). Dus niet alleen hebben deze koele, minder massieve gasreuzen statistisch gezien een metaalrijkere atmosfeer dan hun hetere evenknieeën, ze vertonen ook minder sterke verticale menging. Daarentegen vonden we dat de heetste planeten (>1000K) het best te verklaren zijn met 1x zonne-metalliciteit en met sterke verticale menging ($K_{zz}= 10^{12}$~\cmcms). Hogere niveaus van verticale menging in hetere atmosferen zijn theoretisch voorspeld en waargenomen in de atmosferen van bruine dwergen. Ons resultaat ondersteunt deze beide studies.

In ons onderzoek naar planeten die voor hun ster langs gaan, vonden we geen duidelijke trends in de transmissiemetriek met toenemende temperatuur voor de ultrahete planeten, hoewel we slechts een paar ultrahete Jupiters in onze steekproef hebben. In Hoofdstuk \ref{eclipses} breiden we ons onderzoek uit naar de emissie van de dagzijden van exoplaneten. We hebben een onderzoek van 78 planeten met secundaire eclipsen gebruikt om te bekijken hoe de abundantie van \ce{CH4} en CO zich manifesteert aan de dagzijde. Dit onderzoek bevatte twee keer zoveel hete planeten als ons transmissie-onderzoek, en slechts enkele daarvan vallen binnen het verwachte temperatuurbereik waarin de atmosfeer door methaan zou worden gedomineerd. Bovendien dringt emissiefotometrie dieper door in de atmosfeer (tot het ~1-10 bar drukniveau) dan transmissiefotometrie (~1mbar drukniveau), waardoor we hogere temperaturen en daarmee waarschijnlijk een lagere methaanabundantie zien.

We hebben deze steekproef van planeten onderzocht door de 3,6 en 4,5~\um~ helderheidstemperaturen van de dagzijden van de planeten te berekenen. We hebben ook een metriek gedefinieerd, die we de afwijking van een zwarte straler noemen, die de emissie of absorptie in de 4,5~\um bandfilter meet ten opzichte van de 3,6~\um~filter. De 4,5 \um filter meet de CO-functie en de 3,6 \um filter meet dicht bij het continuüm door het gebrek aan methaan in de hetere planeten. Daardoor stelt deze metriek ons in staat het temperatuur-drukprofiel te bepalen door de sterkte van de relatieve CO emissie of absorptie te meten. Wij hebben een overgang gevonden in de afwijking van een zwarte straler tussen de hete en de ultrahete Jupiters rond 1700K, waar de hetere planeten een sterker teken van CO in de emissie bleken te hebben. We hebben onderzocht wat de oorzaak hiervan kan zijn door het resultaat te vergelijken met een nieuw raster van zelfconsistente 1D radiatieve en convectieve modellen, variërend in metalliciteit, koolstof-zuurstofverhouding (C/O), oppervlaktezwaartekracht en effectieve stellaire temperatuur, waarbij we ervoor gezorgd hebben dat de relevante fysica voor het ontstaan van temperatuurinversies wordt meegenomen. De data zijn opmerkelijk goed in overeenstemming met deze modellen. Wij stellen voor dat de overgang tussen hete en ultrahete Jupiters een statistisch bewijs is van temperatuurinversies bij de heetste planeten, naast de verwachte verschuiving van de Planck-functie.

Hoofdstuk \ref{eclipses} laat zien dat een zorgvuldige berekening van de helderheidstemperaturen en de effectieve temperatuur cruciaal is. We vonden dat het niet integreren over de Spitzer bandfilter of het benaderen van de ster met een zwarte straler in plaats van een PHOENIX model bij het berekenen van de helderheidstemperaturen een vertekening in de resultaten kan veroorzaken. Deze vertekening resulteerde in een verhoging van de gemeten effectieve temperatuur van de planeet vergeleken met de voorspelling van de evenwichtstemperatuur, en de vertekening was sterker voor de planeten rond hetere sterren. Deze onevenredig hogere effectieve temperaturen bij hetere exoplaneten kunnen verkeerd worden geïnterpreteerd als een lagere efficiëntie van atmosferische herdistributie bij de heetste planeten, zoals in eerdere studies is waargenomen. Een andere bron van vertekening komt voort uit het feit dat de effectieve temperatuur gewoonlijk wordt berekend door de spectrale energieverdeling van de planeet te benaderen met een zwarte straler. Er zijn echter slechts twee fotometrische punten, waarvan er één een sterk CO-emissie-teken heeft (4,5~\um~) in het geval van ultrahete Jupiters. In zo'n situatie vertekent de berekening van de effectieve temperatuur als een gewogen gemiddelde van de twee Spitzer-helderheidstemperaturen ook de metingen van de effectieve temperatuur in de richting van hogere temperaturen. Na correctie voor al deze effecten vonden we geen statistisch significante trend in de effectieve temperatuur als functie van de evenwichtstemperatuur. Deze bevinding ondersteunt eerdere beweringen van lagere herverdelingsefficiëntie in hetere planeten niet. Er bleef echter een grote spreiding in de helderheidstemperaturen van hetere planeten vergeleken met koelere planeten, wat suggereerde dat er verschillende herverdelingsefficiënties bestaan voor de heetste planeten.

In Hoofdstuk \ref{w18b} hebben we 10 secundaire eclipsen van de zeer hete Jupiter WASP-18b geanalyseerd en hebben we periodieke variabiliteit gevonden in de 4,5~\um~helderheid van de planeet als functie van de tijd. Met behulp van een sinusoïdaal model hebben we een variabiliteitsperiode van 23,12 $\pm$ 1,66 dagen en een piek-tot-dal amplitude van 456 $\pm$ 71~ppm gevonden, wat overeenkomt met een veranderlijkheid van $\sim$12\%. We hebben mogelijke fysische processen besproken die tot zulke variabiliteit zouden kunnen leiden: koppeling van magnetische velden, variabele windsnelheden, wolken, veranderingen in chemische samenstelling, en we hebben de hypothese dat dit te wijten was aan stellaire variabiliteit uitgesloten. Tenslotte hebben wij onderzocht of dit kan worden gedetecteerd met de huidige geavanceerde instrumenten (HST, TESS) en hebben we vastgesteld dat deze niet de vereiste nauwkeurigheid hebben en dat we moeten kijken naar toekomstige missies om deze variabiliteitsmetingen op te volgen.

In Hoofdstuk \ref{TTVs} we hebben 48 overgangen van sommige van de koelste planeten geobserveerd door Spitzer/IRAC geanalyseerd. We hebben de overgangstijden gemeten van zes planeten uit drie multi-planeet systemen (Kepler-9, Kepler-18 en Kepler-32) en hebben deze tijden vergeleken met de voorspellingen van de Kepler waarnemingen. De onzekerheden van de Spitzer-transits bleken vrij groot te zijn, zodat de resultaten consistent waren met de voorspellingen, maar niet nauwkeurig genoeg om de modellen verder te beperken. Daarnaast hebben we twee planeetovergangen van de circumbinaire planeet Kepler-16b geanalyseerd. We gebruiken deze overgangen in combinatie met archiefgegevens van Kepler om een fotodynamisch model te construeren en we geven updates van de baanelementen van het systeem.

Het werk dat we in deze dissertatie presenteren, heeft het potentieel om als ijkpunt te dienen voor infraroodstudies in de toekomst van exoplaneetwetenschappen, in het bijzonder met de aanstaande James Webb Space Telescope. De waargenomen trends en de effecten die we hebben beschreven in de gerapporteerde studies (verticale menging, temperatuurinversies, wolken, variabiliteit, en complexe dynamica) zullen nog duidelijker worden met de toegenomen precisie, en belangrijk zijn om mee te nemen in toekomstige inspanningen om atmosferen te modelleren. Ons onderzoek draagt bij tot een beter begrip van de atmosfeer van planeten in een brede context, en kan zo een verhelderend licht werpen op de vorming en evolutie van planeten en, uiteindelijk, op de bewoonbaarheid ervan.



% Reset language
\selectlanguage{english}
\renewcommand\chapterautorefname{chapter}%


%%% Local Variables:
%%% mode: latex
%%% TeX-master: "../thesis_renzo"
%%% End:
